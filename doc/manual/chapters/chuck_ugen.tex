\chapter{Unit Generators}

Unit generators (ugens) can be connected using the ChucK operator (~\chuckop)

\example
\begin{verbatim}
        adc => dac;
\end{verbatim}

the above connects the ugen `adc' (a/d convertor, or audio input) to `dac'  
(d/a convertor, or audio output).

Ugens can also unlinked (using =\textless) and relinked (see examples/unchuck.ck). 

A unit generator may have 0 or more control parameters.  
A Ugen's parameters can be set also using the ChucK operator (\chuckop, or -\textgreater) 

\example

\begin{verbatim}
        //connect sine oscillator to dac
        SinOsc osc => dac;
        // set the Osc's frequency to 60.0 hz
        60.0 => osc.freq;
\end{verbatim}
  (see examples/osc.ck)

\newpage

All ugen's have at least the following four parameters:

\control
\begin{chuckitemize}
\item {\bf .gain} - (float, READ/WRITE) - {\it set gain} 
\item {\bf .op} - (int, READ/WRITE) - {\it set operation type }
  \begin{chuckitemize}
    \item 0: stop - always output 0 
    \item 1: normal operation, add all inputs (default)
    \item 2: normal operation, subtract all inputs starting from the earliest connected
    \item 3: normal operation, multiply all inputs
    \item 4: 4 : normal operation, divide inputs starting from the earlist connected
    \item -1: passthru - all inputs to the ugen are summed and passed directly to output
  \end{chuckitemize}
\item {\bf .last} - (float, READ/WRITE) - {\it the last sample computed by the unit generator}
\item {\bf .channels} - (int, READ only) - {\it the number channels on the UGen}
\item {\bf .chan} - (int) - {\it returns a reference on a channel (0 -\textgreater N-1)}
\end{chuckitemize}

Multichannel UGens are adc, dac, Pan2, Mix2

\example
\begin{verbatim}
    Pan2 p;
    // assumes you called chuck with at least --chan5 or -c5 
    p.chan(1) => dac.chan(4);
\end{verbatim}

\newpage
%\bigugenheading{Standard Ugens}
\ugenheading{audio output}

\chuckugen{
\ugen{ dac}
\begin{chuckitemize}
\item digital/analog converter 
\item abstraction for underlying audio output device 
\end{chuckitemize}

\control 
\begin{chuckitemize}
\item {\bf .left} - (UGen) - {\it input to left channel}
\item {\bf .right} - (UGen) - {\it input to right channel}
\item {\bf .chan( int n )} - (UGen) - {\it input to channel N on the device (0 -\textgreater N-1)}
\item {\bf .channels} - (int, READ only) - {\it returns the number of channels open on device}
\end{chuckitemize}
}

\chuckugen{
\ugen{ adc}
\begin{chuckitemize}
\item analog/digital converter 
\item abstraction for underlying audio input device 
\end{chuckitemize}

\control
\begin{chuckitemize}
\item {\bf .left} - (UGen) - {\it output to left channel}
\item {\bf .right} - (UGen) - {\it output to right channel}
\item {\bf .chan( int n )} - (UGen) - {\it output to channel N on the device (0 -\textgreater N-1)}
\item {\bf .channels} - (int, READ only) - {\it returns the number of channels open on device}
\end{chuckitemize}

}

\chuckugen{
\ugen{ blackhole}
\begin{chuckitemize}
\item sample rate sample sucker
\item ( like dac, ticks ugens, but no more ) 
\item see examples/pwm.ck 
\end{chuckitemize}
}

\chuckugen{
\ugen{ Gain}
\begin{chuckitemize}
\item gain control 
\item (NOTE - all unit generators can themselves change their gain) 
\item (this is a way to add N outputs together and scale them) 
\item used in examples/i-robot.ck 
\end{chuckitemize}

\control
\begin{chuckitemize}
\item {\bf .gain} - (float, READ/WRITE) - {\it set gain} ( all ugen's have this)
\end{chuckitemize}
\example
\verbatiminput{examples/gain_example.ck}
}

\ugenheading{wave forms}

\chuckugen{
\ugen{ Noise}
\begin{chuckitemize}
\item white noise generator
\item see examples/noise.ck examples/powerup.ck
\end{chuckitemize}
}

\chuckugen{
\ugen{ Impulse}
\begin{chuckitemize}
\item pulse generator - can set the value of the next sample
\item default for each sample is 0 if not set
\end{chuckitemize}
\control
\begin{chuckitemize}
\item {\bf .next} - (float, READ/WRITE) - {\it set value of next sample}
\end{chuckitemize}
\example
\chuckexample{\verbatiminput{examples/impulse_example.ck}}
}

\chuckugen{
\ugen{ Step}
\begin{chuckitemize}
\item step generator - like Impulse, but once a value is set, it is held for all following samples, until value is set again 
\item see examples/step.ck 
\end{chuckitemize}
\control
\begin{chuckitemize}
\item {\bf .value} - (float, READ/WRITE) - {\it set the current value}
\item {\bf .next} - (float, READ/WRITE) - {\it set the step value}
\end{chuckitemize}
\example
\chuckexample{\verbatiminput{examples/step_example.ck}}

}

\ugenheading{basic signal processing}

\chuckugen{
\ugen{ HalfRect}
\begin{chuckitemize}
\item half wave rectifier 
\item for half-wave rectification 
\end{chuckitemize}
}

\chuckugen{
\ugen{ FullRect}
\begin{chuckitemize}
\item full wave rectifier 
\end{chuckitemize}
}


\ugenheading{filters}

\chuckugen{ 
\ugen{BiQuad}
\begin{chuckitemize}
\item BiQuad (two-pole, two-zero) filter class. 
\end{chuckitemize}
\verbatiminput{examples/ugen/BiQuad.txt}
\control
\begin{chuckitemize}
\item {\bf .b2} (float, READ/WRITE) {\it filter coefficient}
\item {\bf .b1} (float, READ/WRITE) {\it filter coefficient}
\item {\bf .b0} (float, READ/WRITE) {\it filter coefficient}
\item {\bf .a2} (float, READ/WRITE) {\it filter coefficient}
\item {\bf .a1} (float, READ/WRITE) {\it filter coefficient}
\item {\bf .a0} (float, READ only) {\it filter coefficient}

\item {\bf .pfreq} (float, READ/WRITE) {\it set resonance frequency (poles)}
\item {\bf .prad} (float, READ/WRITE) {\it pole radius (<= 1 to be stable)}
\item {\bf .zfreq} (float, READ/WRITE) {\it notch frequency}
\item {\bf .zrad} (float, READ/WRITE) {\it zero radius}
\item {\bf .norm} (float, READ/WRITE) {\it normalization}
\item {\bf .eqzs} (float, READ/WRITE) {\it equal gain zeroes}
\end{chuckitemize}
}


\chuckugen{ 
\ugen{ OnePole}
\begin{chuckitemize}
\item STK one-pole filter class. 
\end{chuckitemize}
\verbatiminput{examples/ugen/OnePole.txt}
\control
\begin{chuckitemize}
\item {\bf .a1} (float, READ/WRITE) {\it filter coefficient}
\item {\bf .b0} (float, READ/WRITE) {\it filter coefficient}
\item {\bf .pole} (float, READ/WRITE) {\it set pole position along real axis of z-plane}
\end{chuckitemize}
}


\chuckugen{ 
\ugen{ TwoPole}
\begin{chuckitemize}
\item STK two-pole filter class. 
\item see examples/powerup.ck 
\end{chuckitemize}
\verbatiminput{examples/ugen/TwoPole.txt}
\control
\begin{chuckitemize}
\item {\bf .a1} (float, READ/WRITE) {\it filter coefficient}
\item {\bf .a2} (float, READ/WRITE) {\it filter coefficient}
\item {\bf .b0} (float, READ/WRITE) {\it  filter coefficient}
\item {\bf .freq} (float, READ/WRITE) {\it filter resonance frequency}
\item {\bf .radius} (float, READ/WRITE) {\it filter resonance radius}
\item {\bf .norm} (float, READ/WRITE) {\it toggle filter normalization}
\end{chuckitemize}
}


\chuckugen{ 
\ugen{ OneZero}
\begin{chuckitemize}
\item STK one-zero filter class. 
\end{chuckitemize}
\verbatiminput{examples/ugen/OneZero.txt}
\control
\begin{chuckitemize}
\item {\bf .zero} (float, READ/WRITE) {\it  set zero position}
\item {\bf .b0} (float, READ/WRITE) {\it filter coefficient}
\item {\bf .b1} (float, READ/WRITE) {\it filter coefficient}
\end{chuckitemize}
}


\chuckugen{ 
\ugen{ TwoZero}
\begin{chuckitemize}
\item STK two-zero filter class. 
\end{chuckitemize}
\verbatiminput{examples/ugen/TwoZero.txt}
\control
\begin{chuckitemize}
\item {\bf .b0} (float, READ/WRITE) {\it filter coefficient}
\item {\bf .b1} (float, READ/WRITE) {\it filter coefficient}
\item {\bf .b2} (float, READ/WRITE) {\it  filter coefficient}
\item {\bf .freq} (float, READ/WRITE) {\it filter notch frequency}
\item {\bf .radius} (float, READ/WRITE) {\it filter notch radius}
\end{chuckitemize}
}


\chuckugen{ 
\ugen{ PoleZero}
\begin{chuckitemize}
\item STK one-pole, one-zero filter class. 
\end{chuckitemize}
\verbatiminput{examples/ugen/PoleZero.txt}
\control
\begin{chuckitemize}
\item {\bf .a1} (float, READ/WRITE) {\it filter coefficient}
\item {\bf .b0} (float, READ/WRITE) {\it filter coefficient}
\item {\bf .b1} (float, READ/WRITE) {\it  filter coefficient}
\item {\bf .blockZero} (float, READ/WRITE) {\it DC blocking filter with given pole position}
\item {\bf .allpass} (float, READ/WRITE) {\it allpass filter with given coefficient}
\end{chuckitemize}
}


\chuckugen{ 
\ugen{ Filter}
\begin{chuckitemize}
\item STK filter class. 
\end{chuckitemize}
\verbatiminput{examples/ugen/Filter.txt}
\control
\begin{chuckitemize}
\item {\bf .coefs} (string, WRITE only) 
\end{chuckitemize}
}


\chuckugen{ 
\ugen{ LPF}
\begin{chuckitemize}
\item Resonant low pass filter.  2nd order Butterworth. (In the future, this class may be expanded so that order and type of filter can be set.)
\end{chuckitemize}
extends FilterBasic

%\verbatiminput{examples/ugen/Filter.txt}
\control
\begin{chuckitemize}
\item {\bf .freq} (float, READ/WRITE) {\it cutoff frequency (Hz)} 
\item {\bf .Q} (float, READ/WRITE) {\it resonance (default is 1)} 
\item {\bf .set} (float, float WRITE only) {\it set freq and Q} 
\end{chuckitemize}
}

\chuckugen{ 
\ugen{ HPF}
\begin{chuckitemize}
\item Resonant high pass filter.  2nd order Butterworth. (In the future, this class may be expanded so that order and type of filter can be set.)
\end{chuckitemize}
extends FilterBasic

%\verbatiminput{examples/ugen/Filter.txt}
\control
\begin{chuckitemize}
\item {\bf .freq} (float, READ/WRITE) {\it cutoff frequency (Hz)} 
\item {\bf .Q} (float, READ/WRITE) {\it resonance (default is 1)} 
\item {\bf .set} (float, float WRITE only) {\it set freq and Q} 
\end{chuckitemize}
}

\chuckugen{ 
\ugen{ BPF}
\begin{chuckitemize}
\item Band pass filter.  2nd order Butterworth. (In the future, this class may be expanded so that order and type of filter can be set.)
\end{chuckitemize}
extends FilterBasic

%\verbatiminput{examples/ugen/Filter.txt}
\control
\begin{chuckitemize}
\item {\bf .freq} (float, READ/WRITE) {\it center frequency (Hz)} 
\item {\bf .Q} (float, READ/WRITE) {\it Q (default is 1)} 
\item {\bf .set} (float, float WRITE only) {\it set freq and Q} 
\end{chuckitemize}
}

\chuckugen{ 
\ugen{ BRF}
\begin{chuckitemize}
\item Band reject filter.  2nd order Butterworth. (In the future, this class may be expanded so that order and type of filter can be set.)
\end{chuckitemize}
extends FilterBasic

%\verbatiminput{examples/ugen/Filter.txt}
\control
\begin{chuckitemize}
\item {\bf .freq} (float, READ/WRITE) {\it center frequency (Hz)} 
\item {\bf .Q} (float, READ/WRITE) {\it Q (default is 1)} 
\item {\bf .set} (float, float WRITE only) {\it set freq and Q} 
\end{chuckitemize}
}

\chuckugen{ 
\ugen{ ResonZ}
\begin{chuckitemize}
\item Resonance filter.  Same as BiQuad with equal gain zeros.
\end{chuckitemize}
extends FilterBasic

%\verbatiminput{examples/ugen/Filter.txt}
\control
\begin{chuckitemize}
\item {\bf .freq} (float, READ/WRITE) {\it center frequency (Hz)} 
\item {\bf .Q} (float, READ/WRITE) {\it Q (default is 1)} 
\item {\bf .set} (float, float WRITE only) {\it set freq and Q} 
\end{chuckitemize}
}

\chuckugen{ 
\ugen{ FilterBasic}
\begin{chuckitemize}
\item base class, don't instantiate.
\end{chuckitemize}
%\verbatiminput{examples/ugen/Filter.txt}
\control
\begin{chuckitemize}
\item {\bf .freq} (float, READ/WRITE) {\it cutoff/center frequency (Hz)} 
\item {\bf .Q} (float, READ/WRITE) {\it resonance/Q} 
\item {\bf .set} (float, float WRITE only) {\it set freq and Q} 
\end{chuckitemize}
}

%\ugenheading{delays}
%
%\chuckugen{
%\ugen{ DelayP}
%\begin{chuckitemize}
%\item variable write delay
%\end{chuckitemize}
%\control
%\begin{chuckitemize}
%\item {\bf .delay} - (dur, READ/WRITE) - {\it delay before subsequent values emerge }
%\item {\bf .window} - (dur, READ/WRITE) - {\it time for 'write head' to move}
%\item {\bf .max} - (dur, READ/WRITE) - {\it max delay possible. trashes buffer, so do it first!}
%\end{chuckitemize}
%}

\ugenheading{sound files}

\chuckugen{
\ugen{ SndBuf}
\begin{chuckitemize}
\item sound buffer ( now interpolating ) 
\item reads from a variety of file formats 
\item see examples/sndbuf.ck 
\end{chuckitemize}
\control
\begin{chuckitemize}
\item {\bf .read} - (string, WRITE only) - {\it loads file for reading}
\item {\bf .chunks} - (int, READ/WRITE) - {\it size of chunk (\# of frames) to read on-demand; 0 implies entire file, default; must be set before reading to take effect.}
\item {\bf .write} - (string, WRITE only) - {\it loads a file for writing (currently unimplemented)}
\item {\bf .pos} - (int, READ/WRITE) - {\it set position (0 \textless p \textless .samples)}
\item {\bf .valueAt} - (int, READ only) - {\it returns the value at sample index}
\item {\bf .loop} - (int, READ/WRITE) - {\it toggle looping }
\item {\bf .interp} - (int, READ/WRITE) - {\it set interpolation (0=drop, 1=linear, 2=sinc)}
\item {\bf .rate} - (float, READ/WRITE) - {\it playback rate (relative to the file's natural speed)}
\item {\bf .play} - (float, READ/WRITE) - {\it play (same as rate) }
\item {\bf .freq} - (float, READ/WRITE) - {\it playback rate (file loops/second)}
\item {\bf .phase} - (float, READ/WRITE) - {\it set phase position (0-1)}
\item {\bf .channel} - (int, READ/WRITE) - {\it select channel (0 \textless x \textless .channels) }
\item {\bf .phaseOffset} - (float, READ/WRITE) - {\it set a phase offset}
\item {\bf .samples} - (int, READ only) - {\it fetch number of samples}
\item {\bf .length} - (dur, READ only) - {\it fetch length as duration}
\item {\bf .channels} - (int, READ only) - {\it fetch number of channels}
\end{chuckitemize}
}

\ugenheading{oscillators}


\chuckugen{ 
\ugen{ Phasor}
\begin{chuckitemize}
\item simple ramp generator ( 0 to 1 ) 
\item this can be fed into other oscillators ( with sync mode of 2 ) 
\item as a phase control. see examples/sixty.ck for an example 
\end{chuckitemize}
\control
\begin{chuckitemize}
\item {\bf .freq}  (float, READ/WRITE)  {\it oscillator frequency (Hz)}
\item {\bf .sfreq}  (float, READ/WRITE)  {\it oscillator frequency (Hz), phase-matched}
\item {\bf .phase}  (float, READ/WRITE)  {\it current phase}
\item {\bf .sync}  (int, READ/WRITE)  {\it (0) sync frequency to input, (1) sync phase to input, (2) fm synth}
\item {\bf .width}  (float, READ/WRITE)  {\it set duration of the ramp in each cycle. (default 1.0)}
\end{chuckitemize}
}

\chuckugen{ 
\ugen{SinOsc}
\begin{chuckitemize}
\item sine oscillator  
\item (see examples/osc.ck)  
\end{chuckitemize}
\control
\begin{chuckitemize}
\item {\bf .freq}  (float, READ/WRITE)  {\it oscillator frequency (Hz)}
\item {\bf .sfreq}  (float, READ/WRITE)  {\it oscillator frequency (Hz), phase-matched}
\item {\bf .phase}  (float, READ/WRITE)  {\it current phase}
\item {\bf .sync}  (int, READ/WRITE) {\it (0) sync frequency to input, (1) sync phase to input, (2) fm synth}
\end{chuckitemize}
}

\chuckugen{
\ugen{ Blit}
\begin{chuckitemize}
\item band limited sine wave oscillator
\end{chuckitemize}
\control
\begin{chuckitemize}
\item {\bf .freq} (float, READ/WRITE) {\it oscillator frequency (Hz)}
\item {\bf .phase} (float, READ/WRITE) {\it current phase}
\item {\bf .harmonics} (int, READ/WRITE) {\it number of harmonics}
\end{chuckitemize}
\example
\chuckexample{\verbatiminput{examples/impulse_example.ck}}
}

\chuckugen{ 
\ugen{ PulseOsc}
\begin{chuckitemize}
\item pulse oscillators 
\item a pulse wave oscillator with variable width. 
\end{chuckitemize}
\control
\begin{chuckitemize}
\item {\bf .freq}  (float, READ/WRITE)  {\it oscillator frequency (Hz)}
\item {\bf .sfreq}  (float, READ/WRITE)  {\it oscillator frequency (Hz), phase-matched}
\item {\bf .phase}  (float, READ/WRITE)  {\it current phase}
\item {\bf .sync}  (int, READ/WRITE) {\it (0) sync frequency to input, (1) sync phase to input, (2) fm synth}
\item {\bf .width}  (float, WRITE)  {\it length of duty cycle (0 - 1)}
\end{chuckitemize}
}

\chuckugen{ 
\ugen{ SqrOsc}
\begin{chuckitemize}
\item square wave oscillator 
\end{chuckitemize}
\control
\begin{chuckitemize}
\item {\bf .freq}  (float, READ/WRITE)  {\it oscillator frequency (Hz)}
\item {\bf .sfreq}  (float, READ/WRITE)  {\it oscillator frequency (Hz), phase-matched}
\item {\bf .phase}  (float, READ/WRITE)  {\it current phase}
\item {\bf .sync}  (int, READ/WRITE)  {\it (0) sync frequency to input, (1) sync phase to input, (2) fm synth}
\item {\bf .width}  (float, WRITE)  {\it fixed duty cycle of 0.5}
\end{chuckitemize}
}

\chuckugen{
\ugen{BlitSquare}
\begin{chuckitemize}
\item band limited square wave oscillator
\end{chuckitemize}
\control
\begin{chuckitemize}
\item {\bf .freq} (float, READ/WRITE) {\it oscillator frequency (Hz)}
\item {\bf .phase} (float, READ/WRITE) {\it current phase}
\item {\bf .harmonics} (int, READ/WRITE) {\it number of harmonics}
\end{chuckitemize}
}

\chuckugen{ 
\ugen{ TriOsc}
\begin{chuckitemize}
\item triangle wave oscillator 
\end{chuckitemize}
\control
\begin{chuckitemize}
\item {\bf .freq}  (float, READ/WRITE)  {\it oscillator frequency (Hz)}
\item {\bf .sfreq}  (float, READ/WRITE)  {\it oscillator frequency (Hz), phase-matched}
\item {\bf .phase}  (float, READ/WRITE)  {\it current phase}
\item {\bf .sync}  (int, READ/WRITE) {\it (0) sync frequency to input, (1) sync phase to input, (2) fm synth}
\item {\bf .width}  (float, READ/WRITE)  {\it control midpoint of triangle (0 - 1)}
\end{chuckitemize}
}

\chuckugen{ 
\ugen{ SawOsc}
\begin{chuckitemize}
\item sawtooth wave oscillator ( triangle, width forced to 0.0 or 1.0 ) 
\end{chuckitemize}
\control
\begin{chuckitemize}
\item {\bf .freq}  (float, READ/WRITE)  {\it oscillator frequency (Hz)}
\item {\bf .sfreq}  (float, READ/WRITE)  {\it oscillator frequency (Hz), phase-matched}
%\item {\bf .phase\underline{ }offset}  (float, READ/WRITE)  {\it phase offset}
\item {\bf .phase}  (float, READ/WRITE)  {\it current phase}
\item {\bf .sync}  (int, READ/WRITE) {\it (0) sync frequency to input, (1) sync phase to input, (2) fm synth}
\item {\bf .width}  (float, READ/WRITE)  {\it increasing ( w \textgreater 0.5 ) or decreasing ( w \textless 0.5 ) }
\end{chuckitemize}
}


\chuckugen{ 
\ugen{ BlitSaw}
\begin{chuckitemize}
\item band limited sawtooth wave oscillator
\end{chuckitemize}
\control
\begin{chuckitemize}
\item {\bf .freq}  (float, READ/WRITE)  {\it oscillator frequency (Hz)}
\item {\bf .phase}  (float, READ/WRITE)  {\it current phase}
\item {\bf .harmonics}  (int, READ/WRITE) {\it number of harmonics}
\end{chuckitemize}
}




\ugenheading{network}

\chuckugen{ 
\ugen{ netout}
\begin{chuckitemize}
\item UDP-based network audio transmitter 
\end{chuckitemize}
\control
\begin{chuckitemize}
\item {\bf .addr} (string, READ/WRITE) {\it target address}
\item {\bf .port} (int, READ/WRITE) {\it target port}
\item {\bf .size} (int, READ/WRITE) {\it packet size} 
\item {\bf .name} (string, READ/WRITE) {\it name}
\end{chuckitemize}
}

\chuckugen{ 
\ugen{ netin}
\begin{chuckitemize}
\item UDP-based network audio receiver 
\end{chuckitemize}
\control
\begin{chuckitemize}
\item {\bf .port} (int, READ/WRITE) {\it set port to receive}
\item {\bf .name} (string, READ/WRITE) {\it name}
\end{chuckitemize}
}

\ugenheading{mono \textless -~ -\textgreater stereo}

\chuckugen{
\ugen{Pan2}
\begin{chuckitemize}
\item spread mono signal to stereo
\item see examples/stereo/moe2.ck
\end{chuckitemize}
\control
\begin{chuckitemize}
\item {\bf .left} (UGen) {\it left (mono) channel out}
\item {\bf .right} (UGen) {\it right (mono) channel out}
\item {\bf .pan} (float, READ/WRITE) {\it pan location value (-1 to 1)}
\end{chuckitemize}
}


\chuckugen{
\ugen{Mix2}
\begin{chuckitemize}
\item mix stereo input down to mono channel
\end{chuckitemize}
\control
\begin{chuckitemize}
\item {.left} - (UGen) {\it left (mono) channel in}
\item {\bf .right} - (UGen) {\it right (mono) channel in}
\item {\bf .pan} - (float, READ/WRITE) {\it mix parameter value (0 - 1)}
\end{chuckitemize}
}





\newpage
%\bigugenheading{STK}
\ugenheading{STK - Instruments}

\chuckugen{
\ugen{StkInstrument \label{ugen_STK}}
(Imported from Instrmnt)
\begin{chuckitemize}
\item Super-class for STK instruments
\end{chuckitemize}
\verbatiminput{examples/ugen/STKinstrument.txt} 
\control
\begin{chuckitemize}
\item {\bf .noteOn} - (float velocity) - {\it trigger note on}
\item {\bf .noteOff} - (float velocity) - {\it trigger note off}
\item {\bf .freq} - (float frequency) - {\it set/get frequency (Hz)}
\item {\bf .controlChange} - (int number, float value) - {\it assert control change - numbers are instrument specific, value range: [0.0 - 128.0] }
\end{chuckitemize}
}



\chuckugen{ 
\ugen{ BandedWG}
\begin{chuckitemize}
\item Banded waveguide modeling class
\end{chuckitemize}
\verbatiminput{examples/ugen/BandedWG.txt}
extends \hypertarget{ugen_STK}{StkInstrument}

\control
\begin{chuckitemize}
\item {\bf .bowPressure} (float, READ/WRITE) {\it bow pressure [0.0 - 1.0] }
\item {\bf .bowMotion} (float, READ/WRITE) {\it bow motion [0.0 - 1.0]}
\item {\bf .bowRate} (float, READ/WRITE) {\it strike Position}
\item {\bf .strikePosition} (float, READ/WRITE) {\it strike Position}
\item {\bf .integrationConstant} - ( float , READ/WRITE ) - {\it ?? [0.0 - 1.0]}
\item {\bf .modesGain} (float, READ/WRITE) {\it amplitude for modes [0.0 - 1.0]}
\item {\bf .preset} (int, READ/WRITE) {\it instrument presets (0 - 3, see above)}
\item {\bf .pluck} (float, READ/WRITE) {\it pluck instrument [0.0 - 1.0]}
\item {\bf .startBowing} (float, READ/WRITE) {\it start bowing [0.0 - 1.0]}
\item {\bf .stopBowing} (float, READ/WRITE) {\it stop bowing [0.0 - 1.0]}
\end{chuckitemize}
{\it (inherited from StkInstrument)}
\begin{chuckitemize}
\item {\bf .noteOn} - (float velocity) - {\it trigger note on}
\item {\bf .noteOff} - (float velocity) - {\it trigger note off}
\item {\bf .freq} - (float frequency) - {\it set/get frequency (Hz)}
\item {\bf .controlChange} - (int number, float value) - {\it assert control change}
\end{chuckitemize}
}


\chuckugen{ 
\ugen{ BlowBotl}
\begin{chuckitemize}
\item STK blown bottle instrument class
\end{chuckitemize}
\verbatiminput{examples/ugen/BlowBotl.txt}
extends \hypertarget{ugen_STK}{StkInstrument}
\control
\begin{chuckitemize}
\item {\bf .noiseGain} - ( float , READ/WRITE ) - {\it noise component gain [0.0 - 1.0]}
\item {\bf .vibratoFreq} - ( float , READ/WRITE ) - {\it vibrato frequency (Hz)}
\item {\bf .vibratoGain} - ( float , READ/WRITE ) - {\it vibrato gain [0.0 - 1.0]}
\item {\bf .volume} - ( float , READ/WRITE ) - {\it yet another volume knob [0.0 - 1.0]}
\item {\bf .startBlowing} (float, READ/WRITE) {\it begin blowing [0.0 - 1.0]}
\item {\bf .stopBlowing} (float, READ/WRITE) {\it stop blowing [0.0 - 1.0]}
\item {\bf .rate} (float, READ/WRITE) - {\it rate of attack (sec)}
\end{chuckitemize}
{\it (inherited from StkInstrument)}
\begin{chuckitemize}
\item {\bf .noteOn} - (float velocity) - {\it trigger note on}
\item {\bf .noteOff} - (float velocity) - {\it trigger note off}
\item {\bf .freq} - (float frequency) - {\it set/get frequency (Hz)}
\item {\bf .controlChange} - (int number, float value) - {\it assert control change }
\end{chuckitemize}
}


\chuckugen{ 
\ugen{ BlowHole}
\begin{chuckitemize}
\item STK clarinet physical model with one register hole and one tonehole. 
\end{chuckitemize}
\verbatiminput{examples/ugen/BlowHole.txt}
extends \hypertarget{ugen_STK}{StkInstrument}
\control
\begin{chuckitemize}
\item {\bf .reed} (float, READ/WRITE) {\it reed stiffness [0.0 - 1.0]}
\item {\bf .noiseGain} - ( float , READ/WRITE ) - {\it noise component gain [0.0 - 1.0]}
\item {\bf .vent} (float, READ/WRITE) {\it vent frequency [0.0 - 1.0]}
\item {\bf .pressure} (float, READ/WRITE) {\it pressure [0.0 - 1.0]}
\item {\bf .tonehole} (float, READ/WRITE) {\it tonehole size [0.0 - 1.0]}
\item {\bf .startBlowing} (float, READ/WRITE) {\it start blowing [0.0 - 1.0]}
\item {\bf .stopBlowing} (float, READ/WRITE) {\it stop blowing [0.0 - 1.0]}
\item {\bf .rate} (float, READ/WRITE) {\it rate of change (sec) }
\end{chuckitemize}
{\it (inherited from StkInstrument)}
\begin{chuckitemize}
\item {\bf .noteOn} - (float velocity) - {\it trigger note on}
\item {\bf .noteOff} - (float velocity) - {\it trigger note off}
\item {\bf .freq} - (float frequency) - {\it set/get frequency (Hz)}
\item {\bf .controlChange} - (int number, float value) - {\it assert control change}
\end{chuckitemize}
}


\chuckugen{ 
\ugen{ Bowed}
\begin{chuckitemize}
\item STK bowed string instrument class.     
\end{chuckitemize}
\verbatiminput{examples/ugen/Bowed.txt}
extends \hypertarget{ugen_STK}{StkInstrument}
\control
\begin{chuckitemize}
\item {\bf .bowPressure} - ( float , READ/WRITE ) - {\it bow pressure [0.0 - 1.0]}
\item {\bf .bowPosition} - ( float , READ/WRITE ) - {\it bow position [0.0 - 1.0]}
\item {\bf .vibratoFreq} - ( float , READ/WRITE ) - {\it vibrato frequency (Hz)}
\item {\bf .vibratoGain} - ( float , READ/WRITE ) - {\it vibrato gain [0.0 - 1.0]}
\item {\bf .volume} - ( float , READ/WRITE ) - {\it volume [0.0 - 1.0]}
\item {\bf .startBowing} (float, READ/WRITE) {\it begin bowing [0.0 - 1.0]}
\item {\bf .stopBowing} (float, READ/WRITE) {\it stop bowing [0.0 - 1.0]}
\item {\bf .rate} (float, READ/WRITE) - {\it rate of attack (sec)}
\end{chuckitemize}
{\it (inherited from StkInstrument)}
\begin{chuckitemize}
\item {\bf .noteOn} - (float velocity) - {\it trigger note on}
\item {\bf .noteOff} - (float velocity) - {\it trigger note off}
\item {\bf .freq} - (float frequency) - {\it set/get frequency (Hz)}
\item {\bf .controlChange} - (int number, float value) - {\it assert control change}
\end{chuckitemize}
}


\chuckugen{ 
\ugen{ Brass}
\begin{chuckitemize}
\item STK simple brass instrument class. 
\end{chuckitemize}
\verbatiminput{examples/ugen/Brass.txt}
extends \hypertarget{ugen_STK}{StkInstrument}
\control
\begin{chuckitemize}
\item {\bf .lip} - ( float , READ/WRITE ) - {\it lip tension [0.0 - 1.0]}
\item {\bf .slide} - ( float , READ/WRITE ) - {\it slide length [0.0 - 1.0]}
\item {\bf .vibratoFreq} - ( float , READ/WRITE ) - {\it vibrato frequency (Hz)}
\item {\bf .vibratoGain} - ( float , READ/WRITE ) - {\it vibrato gain [0.0 - 1.0]}
\item {\bf .volume} - ( float , READ/WRITE ) - {\it volume [0.0 - 1.0]}
\item {\bf .clear} - ( float , WRITE only ) - {\it clear instrument}
\item {\bf .startBlowing} (float, READ/WRITE) {\it start blowing [0.0 - 1.0]}
\item {\bf .stopBlowing} (float, READ/WRITE) {\it stop blowing [0.0 - 1.0]}
\item {\bf .rate} (float, READ/WRITE) {\it rate of change (sec) }
\end{chuckitemize}
{\it (inherited from StkInstrument)}
\begin{chuckitemize}
\item {\bf .noteOn} - (float velocity) - {\it trigger note on}
\item {\bf .noteOff} - (float velocity) - {\it trigger note off}
\item {\bf .freq} - (float frequency) - {\it set/get frequency (Hz)}
\item {\bf .controlChange} - (int number, float value) - {\it assert control change}
\end{chuckitemize}
}


\chuckugen{ 
\ugen{ Clarinet}
\begin{chuckitemize}
\item STK clarinet physical model class. 
\end{chuckitemize}
\verbatiminput{examples/ugen/Clarinet.txt}
extends \hypertarget{ugen_STK}{StkInstrument}
\control
\begin{chuckitemize}
\item {\bf .reed} - ( float , READ/WRITE ) - {\it reed stiffness [0.0 - 1.0]}
\item {\bf .noiseGain} - ( float , READ/WRITE ) - {\it noise component gain [0.0 - 1.0]}
\item {\bf .clear} - ( ) - {\it clear instrument}
\item {\bf .vibratoFreq} - ( float , READ/WRITE ) - {\it vibrato frequency (Hz)}
\item {\bf .vibratoGain} - ( float , READ/WRITE ) - {\it vibrato gain [0.0 - 1.0]}
\item {\bf .pressure} - ( float , READ/WRITE ) - {\it pressure/volume [0.0 - 1.0]}
\item {\bf .startBlowing} - ( float , WRITE only ) - {\it start blowing [0.0 - 1.0]}
\item {\bf .stopBlowing} - ( float , WRITE only ) - {\it stop blowing [0.0 - 1.0]}
\item {\bf .rate} - ( float , READ/WRITE ) - {\it rate of attack (sec)}
\end{chuckitemize}
{\it (inherited from StkInstrument)}
\begin{chuckitemize}
\item {\bf .noteOn} - (float velocity) - {\it trigger note on}
\item {\bf .noteOff} - (float velocity) - {\it trigger note off}
\item {\bf .freq} - (float frequency) - {\it set/get frequency (Hz)}
\item {\bf .controlChange} - (int number, float value) - {\it assert control change}
\end{chuckitemize}
}


\chuckugen{ 
\ugen{ Flute}
\begin{chuckitemize}
\item STK flute physical model class. 
\end{chuckitemize}
\verbatiminput{examples/ugen/Flute.txt}
extends \hypertarget{ugen_STK}{StkInstrument}
\control
\begin{chuckitemize}
\item {\bf .jetDelay} - ( float , READ/WRITE ) - {\it jet delay [...]}
|item {\bf .jetReflection} - ( float , READ/WRITE ) - {\it jet reflection [...]}
\item {\bf .endReflection} - ( float , READ/WRITE ) - {\it end delay [...]}
\item {\bf .noiseGain} - ( float , READ/WRITE ) - {\it noise component gain [0.0 - 1.0]}
\item {\bf .clear} - ( ) - {\it clear instrument}
\item {\bf .vibratoFreq} - ( float , READ/WRITE ) - {\it vibrato frequency (Hz)}
\item {\bf .vibratoGain} - ( float , READ/WRITE ) - {\it vibrato gain [0.0 - 1.0]}
\item {\bf .pressure} - ( float , READ/WRITE ) - {\it pressure/volume [0.0 - 1.0]}
\item {\bf .startBlowing} (float, READ/WRITE) {\it begin bowing [0.0 - 1.0]}
\item {\bf .stopBlowing} (float, READ/WRITE) {\it stop bowing [0.0 - 1.0]}
\item {\bf .rate} (float, READ/WRITE) - {\it rate of attack (sec)}
\end{chuckitemize}
{\it (inherited from StkInstrument)}
\begin{chuckitemize}
\item {\bf .noteOn} - (float velocity) - {\it trigger note on}
\item {\bf .noteOff} - (float velocity) - {\it trigger note off}
\item {\bf .freq} - (float frequency) - {\it set/get frequency (Hz)}
\item {\bf .controlChange} - (int number, float value) - {\it assert control change}
\end{chuckitemize}
}


\chuckugen{ 
\ugen{ Mandolin}
\begin{chuckitemize}
\item STK mandolin instrument model class. 
\item see examples/mand-o-matic.ck 
\end{chuckitemize}
\verbatiminput{examples/ugen/Mandolin.txt}
extends \hypertarget{ugen_STK}{StkInstrument}
\control
\begin{chuckitemize}
\item {\bf .bodySize} (float, READ/WRITE) {\it body size (percentage)}
\item {\bf .pluckPos} (float, READ/WRITE) {\it pluck position [0.0 - 1.0]}
\item {\bf .stringDamping} (float, READ/WRITE) {\it string damping [0.0 - 1.0]}
\item {\bf .stringDetune} (float, READ/WRITE) {\it detuning of string pair [0.0 - 1.0]}
\item {\bf .afterTouch} (float, READ/WRITE) {\it aftertouch (currently unsupported)}
\item {\bf .pluck} - ( float , WRITE only ) - {\it pluck instrument [0.0 - 1.0]}
\end{chuckitemize}
{\it (inherited from StkInstrument)}
\begin{chuckitemize}
\item {\bf .noteOn} - (float velocity) - {\it trigger note on}
\item {\bf .noteOff} - (float velocity) - {\it trigger note off}
\item {\bf .freq} - (float frequency) - {\it set/get frequency (Hz)}
\item {\bf .controlChange} - (int number, float value) - {\it assert control change}
\end{chuckitemize}
}


\chuckugen{ 
\ugen{ ModalBar}
\begin{chuckitemize}
\item STK resonant bar instrument class. 
\item see examples/modalbot.ck 
\end{chuckitemize}
\verbatiminput{examples/ugen/ModalBar.txt}
extends \hypertarget{ugen_STK}{StkInstrument}
\control
\begin{chuckitemize}
\item {\bf .stickHardness} - ( float , READ/WRITE ) - {\it stick hardness [0.0 - 1.0]}
\item {\bf .strikePosition} - ( float , READ/WRITE ) - {\it strike position [0.0 - 1.0]}
\item {\bf .vibratoFreq} - ( float , READ/WRITE ) - {\it vibrato frequency (Hz)}
\item {\bf .vibratoGain} - ( float , READ/WRITE ) - {\it vibrato gain [0.0 - 1.0]}
\item {\bf .directGain} - ( float , READ/WRITE ) - {\it direct gain [0.0 - 1.0]}
\item {\bf .masterGain} - ( float , READ/WRITE ) - {\it master gain [0.0 - 1.0]}
\item {\bf .volume} - ( float , READ/WRITE ) - {\it volume [0.0 - 1.0]}
\item {\bf .preset} - ( int , READ/WRITE ) - {\it choose preset (see above)}
\item {\bf .strike} - ( float , WRITE only ) - {\it strike bar [0.0 - 1.0]}
\item {\bf .damp} - ( float , WRITE only ) - {\it damp bar [0.0 - 1.0]}
\item {\bf .clear} - ( ) - {\it reset [none]}
\item {\bf .mode} - ( int , READ/WRITE ) - {\it select mode [0.0 - 1.0]}
\item {\bf .modeRatio} - ( float , READ/WRITE ) - {\it edit selected mode ratio [...]}
\item {\bf .modeRadius} - ( float , READ/WRITE ) - {\it edit selected mode radius [0.0 - 1.0]}
\item {\bf .modeGain} - ( float , READ/WRITE ) - {\it edit selected mode gain [0.0 - 1.0]}
\end{chuckitemize}
{\it (inherited from StkInstrument)}
\begin{chuckitemize}
\item {\bf .noteOn} - (float velocity) - {\it trigger note on}
\item {\bf .noteOff} - (float velocity) - {\it trigger note off}
\item {\bf .freq} - (float frequency) - {\it set/get frequency (Hz)}
\item {\bf .controlChange} - (int number, float value) - {\it assert control change}
\end{chuckitemize}
}


\chuckugen{ 
\ugen{ Moog}
\begin{chuckitemize}
\item STK moog-like swept filter sampling synthesis class
\item see examples/moogie.ck 
\end{chuckitemize}
\verbatiminput{examples/ugen/Moog.txt}
extends \hypertarget{ugen_STK}{StkInstrument}
\control
\begin{chuckitemize}
\item {\bf .filterQ} - ( float , READ/WRITE ) - {\it filter Q value [0.0 - 1.0]}
\item {\bf .filterSweepRate} - ( float , READ/WRITE ) - {\it filter sweep rate [0.0 - 1.0]}
\item {\bf .vibratoFreq} - ( float , READ/WRITE ) - {\it vibrato frequency (Hz)}
\item {\bf .vibratoGain} - ( float , READ/WRITE ) - {\it vibrato gain [0.0 - 1.0]}
\item {\bf .afterTouch} - ( float , WRITE only ) - {\it aftertouch [0.0 - 1.0]}
\end{chuckitemize}
{\it (inherited from StkInstrument)}
\begin{chuckitemize}
\item {\bf .noteOn} - (float velocity) - {\it trigger note on}
\item {\bf .noteOff} - (float velocity) - {\it trigger note off}
\item {\bf .freq} - (float frequency) - {\it set/get frequency (Hz)}
\item {\bf .controlChange} - (int number, float value) - {\it assert control change}
\end{chuckitemize}
}


\chuckugen{ 
\ugen{ Saxofony}
\begin{chuckitemize}
\item STK faux conical bore reed instrument class.
\end{chuckitemize}
\verbatiminput{examples/ugen/Saxofony.txt}
extends \hypertarget{ugen_STK}{StkInstrument}
\control
\begin{chuckitemize}
\item {\bf .stiffness} - ( float , READ/WRITE ) - {\it reed stiffness [0.0 - 1.0]}
\item {\bf .aperture} - ( float , READ/WRITE ) - {\it reed aperture [0.0 - 1.0]}
\item {\bf .blowPosition} - ( float , READ/WRITE ) - {\it lip stiffness [0.0 - 1.0]}
\item {\bf .noiseGain} - ( float , READ/WRITE ) - {\it noise component gain [0.0 - 1.0]}
\item {\bf .vibratoFreq} - ( float , READ/WRITE ) - {\it vibrato frequency (Hz)}
\item {\bf .vibratoGain} - ( float , READ/WRITE ) - {\it vibrato gain [0.0 - 1.0]}
\item {\bf .clear} - ( ) - {\it clear instrument}
\item {\bf .pressure} - ( float , READ/WRITE ) - {\it pressure/volume [0.0 - 1.0]}
\item {\bf .startBlowing} (float, READ/WRITE) {\it begin blowing [0.0 - 1.0]}
\item {\bf .stopBlowing} (float, READ/WRITE) {\it stop blowing [0.0 - 1.0]}
\item {\bf .rate} (float, READ/WRITE) - {\it rate of attack (sec)}
\end{chuckitemize}
{\it (inherited from StkInstrument)}
\begin{chuckitemize}
\item {\bf .noteOn} - (float velocity) - {\it trigger note on}
\item {\bf .noteOff} - (float velocity) - {\it trigger note off}
\item {\bf .freq} - (float frequency) - {\it set/get frequency (Hz)}
\item {\bf .controlChange} - (int number, float value) - {\it assert control change}
\end{chuckitemize}
}


\chuckugen{ 
\ugen{ Shakers}
\begin{chuckitemize}
\item PhISEM and PhOLIES class. 
\item see examples/shake-o-matic.ck 
\end{chuckitemize}
\verbatiminput{examples/ugen/Shakers.txt}
extends \hypertarget{ugen_STK}{StkInstrument}
\control
\begin{chuckitemize}
\item {\bf .preset} - ( int , READ/WRITE ) - {\it select instrument (0 - 22; see above)}
\item {\bf .energy} - ( float , READ/WRITE ) - {\it shake energy [0.0 - 1.0]}
\item {\bf .decay} - ( float , READ/WRITE ) - {\it system decay [0.0 - 1.0]}
\item {\bf .objects} - ( float , READ/WRITE ) - {\it number of objects [0.0 - 128.0]}
\end{chuckitemize}
{\it (inherited from StkInstrument)}
\begin{chuckitemize}
\item {\bf .noteOn} - (float velocity) - {\it trigger note on}
\item {\bf .noteOff} - (float velocity) - {\it trigger note off}
\item {\bf .freq} - (float frequency) - {\it set/get frequency (Hz)}
\item {\bf .controlChange} - (int number, float value) - {\it assert control change }
\end{chuckitemize}
}


\chuckugen{ 
\ugen{ Sitar}
\begin{chuckitemize}
\item STK sitar string model class. 
\end{chuckitemize}
\verbatiminput{examples/ugen/Sitar.txt}
extends \hypertarget{ugen_STK}{StkInstrument}
\control
\begin{chuckitemize}
\item {\bf .pluck} (float, WRITE only) {\it pluck string [0.0 - 1.0]}
\item {\bf .clear} () {\it reset}
\end{chuckitemize}
{\it (inherited from StkInstrument)}
\begin{chuckitemize}
\item {\bf .noteOn} - (float velocity) - {\it trigger note on}
\item {\bf .noteOff} - (float velocity) - {\it trigger note off}
\item {\bf .freq} - (float frequency) - {\it set/get frequency (Hz)}
\item {\bf .controlChange} - (int number, float value) - {\it assert control change}
\end{chuckitemize}
}

\chuckugen{ 
\ugen{ StifKarp}
\begin{chuckitemize}
\item STK plucked stiff string instrument. 
\item see examples/stifkarp.ck 
\end{chuckitemize}
\verbatiminput{examples/ugen/StifKarp.txt}
extends \hypertarget{ugen_STK}{StkInstrument}
\control
\begin{chuckitemize}
\item {\bf .pickupPosition} - ( float , READ/WRITE ) - {\it pickup position [0.0 - 1.0]}
\item {\bf .sustain} - ( float , READ/WRITE ) - {\it string sustain [0.0 - 1.0]}
\item {\bf .stretch} - ( float , READ/WRITE ) - {\it string stretch [0.0 - 1.0]}
\item {\bf .pluck} - ( float , WRITE only ) - pluck string [0.0 - 1.0]
\item {\bf .baseLoopGain} - ( float , READ/WRITE ) - {\it ?? [0.0 - 1.0]}
\item {\bf .clear} - ( ) - {\it reset instrument}
\end{chuckitemize}
{\it (inherited from StkInstrument)}
\begin{chuckitemize}
\item {\bf .noteOn} - (float velocity) - {\it trigger note on}
\item {\bf .noteOff} - (float velocity) - {\it trigger note off}
\item {\bf .freq} - (float frequency) - {\it set/get frequency (Hz)}
\item {\bf .controlChange} - (int number, float value) - {\it assert control change }
\end{chuckitemize}
}


\chuckugen{ 
\ugen{ VoicForm}
\begin{chuckitemize}
\item Four formant synthesis instrument. 
\item see examples/voic-o-form.ck 
\end{chuckitemize}
\verbatiminput{examples/ugen/VoicForm.txt}
extends \hypertarget{ugen_STK}{StkInstrument}
\control
\begin{chuckitemize}
\item {\bf .phoneme} (string, READ/WRITE) {\it select phoneme ( above )}
\item {\bf .phonemeNum} - ( int , READ/WRITE ) - {\it select phoneme by number [0.0 - 128.0]}
\item {\bf .speak} (float, WRITE only) {\it start singing [0.0 - 1.0]}
\item {\bf .quiet} (float, WRITE only) {\it stop singing [0.0 - 1.0]}
\item {\bf .voiced} (float, READ/WRITE) {\it set mix for voiced component [0.0 - 1.0]}
\item {\bf .unVoiced} (float, READ/WRITE) {\it set mix for unvoiced componenet [0.0 - 1.0]}
\item {\bf .pitchSweepRate} (float, READ/WRITE) {\it pitch sweep [0.0 - 1.0]}
\item {\bf .voiceMix} (float, READ/WRITE) {\it voiced/unvoiced mix [0.0 - 1.0]}
\item {\bf .vibratoFreq} (float, READ/WRITE) {\it vibrato frequency (Hz)}
\item {\bf .vibratoGain} (float, READ/WRITE) {\it  vibrato gain [0.0 - 1.0]}
\item {\bf .loudness} (float, READ/WRITE) {\it  'loudness' of voice [0.0 - 1.0]}
\end{chuckitemize}
{\it (inherited from StkInstrument)}
\begin{chuckitemize}
\item {\bf .noteOn} - (float velocity) - {\it trigger note on}
\item {\bf .noteOff} - (float velocity) - {\it trigger note off}
\item {\bf .freq} - (float frequency) - {\it set/get frequency (Hz)}
\item {\bf .controlChange} - (int number, float value) - {\it assert control change}
\end{chuckitemize}
}

\ugenheading{STK - FM Synths}


\chuckugen{ 
\ugen{ FM \label{ugen_FM}}
\begin{chuckitemize}
\item STK abstract FM synthesis base class
\end{chuckitemize}
\verbatiminput{examples/ugen/FM.txt}
\control
\begin{chuckitemize}
\item {\bf .lfoSpeed} (float, READ/WRITE) {\it modulation speed (Hz)}
\item {\bf .lfoDepth} (float, READ/WRITE) {\it modulation depth [0.0 - 1.0]}
\item {\bf .afterTouch} (float, READ/WRITE) {\it aftertouch [0.0 - 1.0]}
\item {\bf .control1} (float, READ/WRITE) {\it FM control 1 [instrument specific]}
\item {\bf .control2} (float, READ/WRITE) {\it FM control 2 [instrument specific]}
\end{chuckitemize}
{it (inherited from StkInstrument)}
\begin{chuckitemize}
\item {\bf .noteOn} - (float velocity) - {\it trigger note on}
\item {\bf .noteOff} - (float velocity) - {\it trigger note off}
\item {\bf .freq} - (float frequency) - {\it set/get frequency (Hz)}
\item {\bf .controlChange} - (int number, float value) - {\it assert control change}
\end{chuckitemize}
}


\chuckugen{ 
\ugen{ BeeThree}
\begin{chuckitemize}
\item STK Hammond-oid organ FM synthesis instrument. 
\end{chuckitemize}
\verbatiminput{examples/ugen/BeeThree.txt}
extends \hypertarget{ugen_FM}{FM}

\control
\begin{chuckitemize}
\item {\bf (see super classes)}
\end{chuckitemize}
}


\chuckugen{ 
\ugen{ FMVoices}
\begin{chuckitemize}
\item STK singing FM synthesis instrument. 
\end{chuckitemize}
\verbatiminput{examples/ugen/FMVoices.txt}
extends \hypertarget{ugen_FM}{FM}

\control
\begin{chuckitemize}
\item {\bf .vowel} (float, WRITE only) {\it select vowel [0.0 - 1.0]}
\item {\bf .spectralTilt} (float, WRITE only) {\it spectral tilt [0.0 - 1.0]}
\item {\bf .adsrTarget} (float, WRITE only) {\it adsr targets [0.0 - 1.0]}
\end{chuckitemize}
}


\chuckugen{ 
\ugen{ HevyMetl}
\begin{chuckitemize}
\item STK heavy metal FM synthesis instrument. 
\end{chuckitemize}
\verbatiminput{examples/ugen/HevyMetl.txt}
extends \hypertarget{ugen_FM}{FM}

\control
\begin{chuckitemize}
\item {\bf (see super classes)}
\end{chuckitemize}
}


\chuckugen{ 
\ugen{ PercFlut}
\begin{chuckitemize}
\item STK percussive flute FM synthesis instrument. 
\end{chuckitemize}
\verbatiminput{examples/ugen/PercFlut.txt}
extends \hypertarget{ugen_FM}{FM}

\control
\begin{chuckitemize}
\item {\bf (see super classes)}
\end{chuckitemize}
}


\chuckugen{ 
\ugen{ Rhodey}
\begin{chuckitemize}
\item STK Fender Rhodes-like electric piano FM 
\item see examples/rhodey.ck
\end{chuckitemize}
\verbatiminput{examples/ugen/Rhodey.txt}
extends \hypertarget{ugen_FM}{FM}
 
\control
\begin{chuckitemize}
\item {\bf (see super classes)}
\end{chuckitemize}
}


\chuckugen{ 
\ugen{ TubeBell}
\begin{chuckitemize}
\item STK tubular bell (orchestral chime) FM 
\item extends \hypertarget{ugen_FM}{FM}
\end{chuckitemize}
\verbatiminput{examples/ugen/TubeBell.txt}
\control
\begin{chuckitemize}
\item {\bf (see super classes)}
\end{chuckitemize}
}

\chuckugen{ 
\ugen{ Wurley}
\begin{chuckitemize}
\item STK Wurlitzer electric piano FM 
\item see examples/wurley.ck extends 
\end{chuckitemize}
\verbatiminput{examples/ugen/Wurley.txt}
extends \hypertarget{ugen_FM}{FM}

\control
\begin{chuckitemize}
\item {\bf (see super classes)}
\end{chuckitemize}
}

\ugenheading{STK - Delay}

\chuckugen{ 
\ugen{ Delay}
\begin{chuckitemize}
\item STK non-interpolating delay line class
\item see examples/net\_relay.ck 
\end{chuckitemize}
\verbatiminput{examples/ugen/Delay.txt}
\control
\begin{chuckitemize}
\item {\bf .delay} (dur, READ/WRITE) {\it length of delay}
\item {\bf .max} (dur, READ/WRITE) {\it max delay (buffer size)}
\end{chuckitemize}
}

\chuckugen{ 
\ugen{ DelayA}
\begin{chuckitemize}
\item STK allpass interpolating delay line class. 
\end{chuckitemize}
\verbatiminput{examples/ugen/DelayA.txt}
\control
\begin{chuckitemize}
\item {\bf .delay} (dur, READ/WRITE) {\it length of delay}
\item {\bf .max} (dur, READ/WRITE) {\it max delay (buffer size)}
\end{chuckitemize}
}


\chuckugen{ 
\ugen{ DelayL}
\begin{chuckitemize}
\item STK linear interpolating delay line class. 
\item see examples/i-robot.ck 
\end{chuckitemize}
\verbatiminput{examples/ugen/DelayL.txt}
\control
\begin{chuckitemize}
\item {\bf .delay} (dur, READ/WRITE) {\it length of delay}
\item {\bf .max} (dur, READ/WRITE) {\it max delay (buffer size)}
\end{chuckitemize}
}


\chuckugen{ 
\ugen{ Echo}
\begin{chuckitemize}
\item STK echo effect class.
\end{chuckitemize} 
\verbatiminput{examples/ugen/Echo.txt}
\control
\begin{chuckitemize}
\item {\bf .delay} (dur, READ/WRITE) {\it length of echo}
\item {\bf .max} (dur, READ/WRITE) {\it max delay}
\item {\bf .mix} (float, READ/WRITE) {\it mix level (wet/dry)}
\end{chuckitemize}
}


\ugenheading{STK - Envelopes}

\chuckugen{ 
\ugen{ Envelope}
\begin{chuckitemize}
\item STK envelope base class. 
\item see examples/sixty.ck 
\end{chuckitemize}
\verbatiminput{examples/ugen/Envelope.txt}
\control
\begin{chuckitemize}
\item {\bf .keyOn} (int, WRITE only) {\it ramp to 1.0}
\item {\bf .keyOff} (int, WRITE only) {\it ramp to 0.0}
\item {\bf .target} (float, READ/WRITE) {\it ramp to arbitrary value}
\item {\bf .time} (float, READ/WRITE) {\it time to reach target (in second)}
\item {\bf .duration} (dur, READ/WRITE) {\it time to reach target}
\item {\bf .rate} (float, READ/WRITE) {\it rate of change}
\item {\bf .value} (float, READ/WRITE) {\it set immediate value}
\end{chuckitemize}
}


\chuckugen{ 
\ugen{ ADSR}
\begin{chuckitemize}
\item STK ADSR envelope class. 
\item see examples/adsr.ck 
\end{chuckitemize}
\verbatiminput{examples/ugen/ADSR.txt}
\control
\begin{chuckitemize}
\item {\bf .keyOn} (int, WRITE only) {\it start the attack for non-zero values}
\item {\bf .keyOff} (int, WRITE only) {\it start the release for non-zero values}
\item {\bf .attackTime} (dur, WRITE only) {\it attack time}
\item {\bf .attackRate} (float, READ/WRITE) {\it attack rate}
\item {\bf .decayTime} (dur, READ/WRITE) {\it decay}
\item {\bf .decayRate} (float, READ/WRITE) {\it decay rate}
\item {\bf .sustainLevel} (float, READ/WRITE) {\it sustain level}
\item {\bf .releaseTime} (dur, READ/WRITE) {\it release time}
\item {\bf .releaseRate} (float, READ/WRITE) {\it release rate}
\item {\bf .state} (int, READ only) {\it attack=0, decay=1, sustain=2, release=3,done=4}
\end{chuckitemize}
}


\ugenheading{STK - Reverbs}

\chuckugen{ 
\ugen{ JCRev}
\begin{chuckitemize}
\item John Chowning's reverberator class. 
\end{chuckitemize}
\verbatiminput{examples/ugen/JCRev.txt}
\control
\begin{chuckitemize}
\item {\bf .mix} (float, READ/WRITE) {\it mix level}
\end{chuckitemize}
}


\chuckugen{ 
\ugen{ NRev}
\begin{chuckitemize}
\item CCRMA's NRev reverberator class. 
\end{chuckitemize}
\verbatiminput{examples/ugen/NRev.txt}
\control
\begin{chuckitemize}
\item {\bf .mix} (float, READ/WRITE) {\it mix level}
\end{chuckitemize}
}


\chuckugen{ 
\ugen{ PRCRev}
\begin{chuckitemize}
\item Perry's simple reverberator class. 
\end{chuckitemize}
\verbatiminput{examples/ugen/PRCRev.txt}
\control
\begin{chuckitemize}
\item {\bf .mix} (float, READ/WRITE) {\it mix level}
\end{chuckitemize}
}

\ugenheading{STK - Components}

\chuckugen{ 
\ugen{ Chorus}
\begin{chuckitemize}
\item STK chorus effect class. 
\end{chuckitemize}
\verbatiminput{examples/ugen/Chorus.txt}
\control
\begin{chuckitemize}
\item {\bf .modFreq} (float, READ/WRITE) {\it modulation frequency}
\item {\bf .modDepth} (float, READ/WRITE) {\it modulation depth}
\item {\bf .mix} (float, READ/WRITE) {\it effect mix}
\end{chuckitemize}
}


\chuckugen{ 
\ugen{ Modulate}
\begin{chuckitemize}
\item STK periodic/random modulator. 
\end{chuckitemize}
\verbatiminput{examples/ugen/Modulate.txt}
\control
\begin{chuckitemize}
\item {\bf .vibratoRate} (float, READ/WRITE) {\it set rate of vibrato}
\item {\bf .vibratoGain} (float, READ/WRITE) {\it gain for vibrato}
\item {\bf .randomGain} (float, READ/WRITE) {\it gain for random contribution}
\end{chuckitemize}
}


\chuckugen{ 
\ugen{ PitShift}
\begin{chuckitemize}
\item STK simple pitch shifter effect class. 
\end{chuckitemize}
\verbatiminput{examples/ugen/PitShift.txt}
\control
\begin{chuckitemize}
\item {\bf .mix} (float, READ/WRITE) {\it effect dry/wet mix level}
\item {\bf .shift} (float, READ/WRITE) {\it degree of pitch shifting}
\end{chuckitemize}
}


\chuckugen{ 
\ugen{ SubNoise}
\begin{chuckitemize}
\item STK sub-sampled noise generator. 
\end{chuckitemize}
\verbatiminput{examples/ugen/SubNoise.txt}
\control
\begin{chuckitemize}
\item {\bf .rate} (int, READ/WRITE) {\it subsampling rate}
\end{chuckitemize}
}

\ugenheading{STK - File I/O}

\chuckugen{ 
\ugen{ WvIn}
\begin{chuckitemize}
\item STK audio data input base class. 
\end{chuckitemize}
\verbatiminput{examples/ugen/WvIn.txt}
\control
\begin{chuckitemize}
\item {\bf .rate} (float, READ/WRITE) {\it playback rate}
\item {\bf .path} (string, READ/WRITE) {\it specifies file to be played}
\end{chuckitemize}
}


\chuckugen{ 
\ugen{ WaveLoop}
\begin{chuckitemize}
\item STK waveform oscillator class. 
\item see examples/dope.ck
\end{chuckitemize} 
\verbatiminput{examples/ugen/WvLoop.txt}
\control
\begin{chuckitemize}
\item {\bf .freq} (float, READ/WRITE) {\it frequency of playback (loops/second)}
\item {\bf .addPhase} (float, READ/WRITE) {\it offset by phase}
\item {\bf .addPhaseOffset} (float, READ/WRITE) {\it set phase offset}
\end{chuckitemize}
}


\chuckugen{ 
\ugen{ WvOut}
\begin{chuckitemize}
\item STK audio data output base class. 
\end{chuckitemize}
\verbatiminput{examples/ugen/WvOut.txt}
\control
\begin{chuckitemize}
\item {\bf .matFilename} (string, WRITE only) {\it open a matlab file for writing}
\item {\bf .sndFilename} (string, WRITE only) {\it open snd file for writing}
\item {\bf .wavFilename} (string, WRITE only) {\it open WAVE file for writing}
\item {\bf .rawFilename} (string, WRITE only) {\it open raw file for writing}
\item {\bf .aifFilename} (string, WRITE only) {\it open AIFF file for writing}
\item {\bf .closeFile} () {\it close file properly}
\end{chuckitemize}
}


\ugenheading{events}

\chuckugen{
\ugen{Event}
\begin{chuckitemize}
\item event handler
\end{chuckitemize}
\control
\begin{chuckitemize}
\item {\bf .signal()} {\it signals first waiting shred}
\item {\bf .broadcast()} {\it signals all shreds}

\end{chuckitemize}
}

\chuckugen{
\ugen{MidiIn}
\begin{chuckitemize}
\item MIDI receiver
\end{chuckitemize}
extends Event
\control
\begin{chuckitemize}
\item {\bf .open} (int, READ/WRITE) \textrightarrow bool {\it set port to receive}
\item {\bf .open} (string, READ/WRITE) \textrightarrow bool {\it set port to receive, exact name match or substring}
\item {\bf .recv} (MidiMsg, READ) \textrightarrow bool {\it receives MIDI input, returns false if no data available}
\item {\bf .good} (READ only) \textrightarrow bool {\it check if open}
\item {\bf .num} (READ only) \textrightarrow int {\it port number}
\item {\bf .name} (READ only) \textrightarrow string {\it port name}
\end{chuckitemize}
(see MIDI tutorial)
}

\chuckugen{
\ugen{MidiOut}
\begin{chuckitemize}
\item MIDI sender
\end{chuckitemize}
extends Event
\control
\begin{chuckitemize}
\item {\bf .open} (int, READ/WRITE) \textrightarrow bool {\it set port to send}
\item {\bf .open} (string, READ/WRITE) \textrightarrow bool {\it set port to send, exact name match or substring}
\item {\bf .send} (MidiMsg, WRITE) {\it sends MIDI output}
\item {\bf .good} (READ only) \textrightarrow bool {\it check if open}
\item {\bf .num} (READ only) \textrightarrow int {\it port number}
\item {\bf .name} (READ only) \textrightarrow string {\it port name}
\end{chuckitemize}
(see MIDI tutorial)
}

\chuckugen{
\ugen{MidiMsg}
\begin{chuckitemize}
\item MIDI data holder
\end{chuckitemize}
\control
\begin{chuckitemize}
\item {\bf .data1} (int, READ/WRITE) {\it first byte of data (member variable)}
\item {\bf .data2} (int, READ/WRITE) {\it second byte of data (member variable)}
\item {\bf .data3} (int, READ/WRITE) {\it third byte of data (member variable)}
\end{chuckitemize}
(see MIDI tutorial)
}



\chuckugen{
\ugen{OscRecv}
\begin{chuckitemize}
\item Open Sound Control receiver
\end{chuckitemize}
\control
\begin{chuckitemize}
\item {\bf .port} (int, READ/WRITE) {\it set port to receive}
\item {\bf .listen} () {\it starts object listening to port}
\item {\bf .listen} (int(port)) {\it starts object listening to port}
\item {\bf .event} (string(spec), READ/WRITE) {\it define string for event to receive in form ``address[,types]''}
\item {\bf .event} (string(name), string(type), READ/WRITE) {\it define string for event to receive}
\item {\bf .stop} () {\it stops listening to port}
\end{chuckitemize}
(see events tutorial)
}

\chuckugen{
\ugen{OscSend}
\begin{chuckitemize}
\item Open Sound Control sender
\end{chuckitemize}
\control
\begin{chuckitemize}
\item {\bf .setHost} (string(host), int(port), WRITE) {\it set port on the host to receive}
\item {\bf .startMsg} (string(spec), WRITE)  {\it start message, spec should be in form ``address[,types...]''}
\item {\bf .startMsg} (string(address), string(types), WRITE)  {\it start message with address and types}
\item {\bf .addInt} (int(value), WRITE)  {\it  append value to message}
\item {\bf .addFloat} (float(value), WRITE)  {\it append value to message}
\item {\bf .addString} (string(value), WRITE)  {\it append value to message}
\item {\bf .openBundle} (time(value), WRITE)  {\it start bundle with the 'immediately' time tag}
\item {\bf .closeBundle} (WRITE)  {\it close current bundle}
\item {\bf .hold} (bool(on), WRITE)  {\it don't send messages immediately, hold until kicked or until a message is finished without this flag}
\item {\bf .kick} (WRITE)  {\it kick currently held message}
\end{chuckitemize}
(see events tutorial)
}

\chuckugen{
\ugen{OscEvent}
\begin{chuckitemize}
\item Open Sound Control event
\end{chuckitemize}
extends Event
\control
\begin{chuckitemize}
\item {\bf .nextMsg} (READ) \textrightarrow bool {\it try to get next message from queue, return false if none left}
\item {\bf .getFloat} (READ) \textrightarrow float {\it gets next value from message as float}
\item {\bf .getInt} (READ) \textrightarrow int {\it gets next value from message as int}
\item {\bf .getString} (READ) \textrightarrow string {\it gets next value from message as string}
\item {\bf .hasMsg} (READ) \textrightarrow bool {\it check if there are messages left in queue}
\end{chuckitemize}
(see events tutorial)
}

\chuckugen{
\ugen{OscOut}
\begin{chuckitemize}
\item New Open Sound Control output
\end{chuckitemize}
\control
\begin{chuckitemize}
\item {\bf .dest} (string(host), int(port), WRITE) {\it set port on the host to receive}
\item {\bf .start} (string(method), WRITE) {\it begin message for method/address}
\item {\bf .start} (string(method), string(host), int(port), WRITE) {\it begin message for method/address and destination}
\item {\bf .add} (int(value), WRITE) {\it append value}
\item {\bf .add} (float(value), WRITE) {\it append value}
\item {\bf .add} (string(value), WRITE) {\it append value}
\item {\bf .send} () {\it finish and send message}
\end{chuckitemize}
(see events tutorial)
}

\chuckugen{
\ugen{OscIn}
\begin{chuckitemize}
\item New Open Sound Control input
\end{chuckitemize}
\control
\begin{chuckitemize}
\item {\bf .port} (int, READ/WRITE) {\it set port to receive}
\item {\bf .addAddress} (string(address), WRITE) {\it add OSC address/spec to receive}
\item {\bf .removeAddress} (string(address), WRITE) {\it remove OSC address/spec to receive}
\item {\bf .removeAllAddresses} () {\it remove all previously added OSC addresses/specs}
\item {\bf .listenAll} () {\it listen for all OSC messages}
\item {\bf .recv} (OscMsg, WRITE) \textrightarrow bool {\it return next message from queue or return false if queue empty}
\end{chuckitemize}
(see events tutorial)
}

\chuckugen{
\ugen{OscMsg}
\begin{chuckitemize}
\item New Open Sound Control message
\end{chuckitemize}
\control
\begin{chuckitemize}
\item {\bf .address} (string READ/WRITE) {\it OSC address}
\item {\bf .typetag} (string READ/WRITE) {\it OSC type tag}
\item {\bf .numArgs} (int READ) {\it count of OSC arguments}
\item {\bf .getInt} (int READ) \textrightarrow int {\it get i'th argument as int}
\item {\bf .getFloat} (int READ) \textrightarrow int {\it get i'th argument as float}
\item {\bf .getString} (int READ) \textrightarrow string {\it get i'th argument as string}
\end{chuckitemize}
(see events tutorial)
}

\chuckugen{
\ugen{KBHit}
\begin{chuckitemize}
\item ascii keyboard event
\end{chuckitemize}
extends Event
\control
\begin{chuckitemize}
\item {\bf .getchar} (int, READ) {\it ascii value}
\item {\bf .more} (int, READ only) {\it returns 1 if multiple keys have been pressed}
\end{chuckitemize}
(see events tutorial)
}


\chuckugen{
\ugen{HidIn}
\begin{chuckitemize}
\item HID receiver
\end{chuckitemize}
extends Event
\control
\begin{chuckitemize}
\item {\bf .openJoystick} (int(which), WRITE only) {\it open joystick number}
\item {\bf .openMouse} (int(which), WRITE only) {\it open mouse number}
\item {\bf .openKeyboard} (int(which), WRITE only) {\it open keyboard number}
\item {\bf .num} () {\it return joystick/mouse/keyboard number}
\item {\bf .recv} (HidMsg, READ only) {\it writes the next message available for this device to the argument}
\item {\bf .name} () {\it return device name}
\end{chuckitemize}
(see events tutorial)
}



\chuckugen{
\ugen{HidMsg}
\begin{chuckitemize}
\item HID data holder
\end{chuckitemize}
\control
\begin{chuckitemize}
\item {\bf .isAxisMotion} (int, READ only) {\it non-zero if this message corresponds to movement of a joystick axis}
\item {\bf .isButtonDown} (int, READ only) {\it non-zero if this message corresponds to button down or key down of any device type}
\item {\bf .isButtonUp} (int, READ only) {\it non-zero if this message corresponds to button up or key up of any device type}
\item {\bf .isMouseMotion} (int, READ only) {\it non-zero if this message corresponds to motion of a pointing device}
\item {\bf .isHatMotion} (int, READ only) {\it non-zero if this message corresponds to motion of a joystick hat, point-of-view switch, or directional pad}
\item {\bf .which} (int, READ/WRITE) {\it HID element number (member variable)}
\item {\bf .axisPosition} (float, READ/WRITE) {\it position of joystick axis in the range [-1.0, 1.0] (member variable)}
\item {\bf .deltaX} (float, READ/WRITE) {\it change in X axis of pointing device (member variable)}
\item {\bf .deltaY} (float, READ/WRITE) {\it change in Y axis of pointing device (member variable)}
\item {\bf .deviceNum} (float, READ/WRITE) {\it device number which produced this message (member variable)}
\item {\bf .deviceType} (float, READ/WRITE) {\it device type which produced this message (member variable)}
\item {\bf .type} (int, READ/WRITE) {\it message/HID element type (member variable)}
\item {\bf .idata} (int, READ/WRITE) {\it data (member variable)}
\item {\bf .fdata} (int, READ/WRITE) {\it data (member variable)}
\end{chuckitemize}
(see events tutorial)
}

